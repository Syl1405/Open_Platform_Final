%Copyright 2014 Jean-Philippe Eisenbarth
%This program is free software: you can 
%redistribute it and/or modify it under the terms of the GNU General Public 
%License as published by the Free Software Foundation, either version 3 of the 
%License, or (at your option) any later version.
%This program is distributed in the hope that it will be useful,but WITHOUT ANY 
%WARRANTY; without even the implied warranty of MERCHANTABILITY or FITNESS FOR A 
%PARTICULAR PURPOSE. See the GNU General Public License for more details.
%You should have received a copy of the GNU General Public License along with 
%this program.  If not, see <http://www.gnu.org/licenses/>.

%Based on the code of Yiannis Lazarides
%http://tex.stackexchange.com/questions/42602/software-requirements-specification-with-latex
%http://tex.stackexchange.com/users/963/yiannis-lazarides
%Also based on the template of Karl E. Wiegers
%http://www.se.rit.edu/~emad/teaching/slides/srs_template_sep14.pdf
%http://karlwiegers.com
\documentclass{scrreprt}
\usepackage{listings}
\usepackage{CJKutf8}
\usepackage{underscore}
\usepackage{array}
\usepackage[bookmarks=true]{hyperref}
\usepackage[utf8]{inputenc}
\usepackage[english]{babel}
\hypersetup{
    bookmarks=false,    % show bookmarks bar?
    pdftitle={Software Requirement Specification},    % title
    pdfauthor={Jean-Philippe Eisenbarth},                     % author
    pdfsubject={TeX and LaTeX},                        % subject of the document
    pdfkeywords={TeX, LaTeX, graphics, images}, % list of keywords
    colorlinks=true,       % false: boxed links; true: colored links
    linkcolor=blue,       % color of internal links
    citecolor=black,       % color of links to bibliography
    filecolor=black,        % color of file links
    urlcolor=purple,        % color of external links
    linktoc=page            % only page is linked
}%
\def\myversion{1.0 }
\date{}
%\title
\usepackage{hyperref}
\begin{document}
\renewcommand\arraystretch{2}
\begin{CJK*}{UTF8}{bsmi}
		

\begin{flushright}
    \rule{16cm}{5pt}\vskip1cm
    \begin{bfseries}
        \Huge{SOFTWARE REQUIREMENTS\\ SPECIFICATION}\\
        \vspace{1.5cm}
        for\\
        \vspace{1.5cm}
        $<$Open PlatForm Software$>$\\
        \vspace{1.5cm}
	
        Prepared by \\
  \vspace{1.5cm}
$<$唐岳,陳昱安,劉紋琦,張藝憲$>$\\
        \vspace{1.5cm}
        $<$Organization$>$\\
        \vspace{1.5cm}
        \today\\
    \end{bfseries}
\end{flushright}

\tableofcontents

\chapter{Introduction}

\section{Purpose}
本程式的目的是建立一個能夠準確分析音訊源的模型,以辨別聲音的種類。 應用面上包括:
\begin{enumerate}
\item 由於本系統可以辨別槍聲,可以拓展至即時監聽市中心的聲音,防範恐怖攻擊
\item 在國道上設置錄音裝置,辨別汽車引擎聲,觀察數值消長以界定是否塞車
\item 設置於嬰兒房,辨識小孩的哭聲,讓家長能更及時照顧到嬰兒的需求
\end{enumerate}
\section{Intended Audience and Reading Suggestions}
此項目是音訊源分析及辨識的模型,對於各方面需要分辨聲音或是用聲音辨識做輔助性功能都很有用。 如:城市的管理者、用路人、家長等

\section{Project Scope}
\begin{enumerate}
\item 盡量允許各種格式的音訊源
\item 限制可辨別的音訊類別為事先定義的十種聲音
\item 完成UI介面
\item 防呆機制與例外處理
\end{enumerate}



\chapter{Overall Description}

\section{Product Perspective}
$<$Describe the context and origin of the product being specified in this SRS.  
For example, state whether this product is a follow-on member of a product 
family, a replacement for certain existing systems, or a new, self-contained 
product. If the SRS defines a component of a larger system, relate the 
requirements of the larger system to the functionality of this software and 
identify interfaces between the two. A simple diagram that shows the major 
components of the overall system, subsystem interconnections, and external 
interfaces can be helpful.$>$

\section{Product Functions}
\begin{enumerate}
\item 註冊與登入
\item 轉換音訊為頻譜圖
\item 顯示頻譜圖片於視窗上
\item 預測音訊/頻譜圖的類別並輸出
\end{enumerate}

\section{User Classes and Characteristics}
\begin{enumerate}
\item 一般使用者: 通過圖形化介面操作,可以輕鬆地完成辨識,分析有興趣的聲音
\item 城市管理者: 使用shell script批次實時監測並進行辨識
\item 家長: 架設錄音裝置,於嬰兒哭鬧時可獲得預警訊息
\end{enumerate}

\section{Operating Environment}
\begin{enumerate}
\item OS: Windows 10
\item Python Runtime: version 3.6
\item Packages: run pip install -r requirements.txt
\item Tensorflow backend
\end{enumerate}

\section{Design and Implementation Constraints}
訓練模型時由於輸入圖片較大(5125121),建議可用GPU RAM在4GB或以上,否則會因頻繁重分配記憶體造成效率低下 由於model已經訓練好,使用者只要在Python環境下,安裝所需套件就可執行

\section{Assumptions and Dependencies}

\begin{center}
\begin{tabular}{|l|l|}
\hline \multicolumn{2}{|c|}{需自行安裝的套件:} \\\hline
pandas  & 讀取train.csv的label  \\ \hline
numpy  & 將原生list轉為更有效率的numpy.array用於訓練模型  \\ \hline
PIL  & 圖片讀取  \\\hline
scipy & 圖片處理 \\\hline
matplotlib  & 圖片與圖表視覺化呈現  \\\hline
librosa  & 音訊分析與處理  \\\hline
sklearn  & 調用one hot encoding, shuffle split  \\\hline
tensorflow  & keras底下引用的深度學習核心  \\\hline
keras  & 深度學習的高階API  \\
\hline \multicolumn{2}{|c|}{Python內建的:} \\\hline
tkinter       & GUI視窗介面  \\ \hline
pickle        & 保存資料  \\ \hline
os  & 讀取檔案  \\\hline

\end{tabular}
\end{center}

\chapter{External Interface Requirements}

\section{User Interfaces}
$<$Describe the logical characteristics of each interface between the software 
product and the users. This may include sample screen images, any GUI standards 
or product family style guides that are to be followed, screen layout 
constraints, standard buttons and functions (e.g., help) that will appear on 
every screen, keyboard shortcuts, error message display standards, and so on.  
Define the software components for which a user interface is needed. Details of 
the user interface design should be documented in a separate user interface 
specification.$>$

\section{Hardware Interfaces}
$<$Describe the logical and physical characteristics of each interface between 
the software product and the hardware components of the system. This may include 
the supported device types, the nature of the data and control interactions 
between the software and the hardware, and communication protocols to be 
used.$>$

\section{Software Interfaces}
$<$Describe the connections between this product and other specific software 
components (name and version), including databases, operating systems, tools, 
libraries, and integrated commercial components. Identify the data items or 
messages coming into the system and going out and describe the purpose of each.  
Describe the services needed and the nature of communications. Refer to 
documents that describe detailed application programming interface protocols.  
Identify data that will be shared across software components. If the data 
sharing mechanism must be implemented in a specific way (for example, use of a 
global data area in a multitasking operating system), specify this as an 
implementation constraint.$>$


\chapter{System Features}

\section{Description and Priority}
概述為使用者如何去註冊,才可以使用這個系統\\
Priority – Medium

\section{Stimulus/Response Sequences}
\begin{center}
\begin{tabular}{|l|l|}\hline
系統反應動作使用 & 使用者操作動作  \\ \hline
 & a.使用者想要使用這系統  \\ \hline
b.系統回應需要先登入 &   \\\hline
 & c.使用者想登入按下登入按鈕  \\\hline
d.系統回應需要先註冊&   \\\hline
\end{tabular}
\end{center}

\section{Functional Requirements}
$<$Itemize the detailed functional requirements associated with this feature.  
These are the software capabilities that must be present in order for the user 
to carry out the services provided by the feature, or to execute the use case.  
Include how the product should respond to anticipated error conditions or 
invalid inputs. Requirements should be concise, complete, unambiguous, 
verifiable, and necessary. Use “TBD” as a placeholder to indicate when necessary 
information is not yet available.$>$

\chapter{Other Nonfunctional Requirements}
\begin{enumerate}
\item 清楚的程式架構及簡單明瞭的註解
\item 跨平台相容性
\item 系統需於每月進行pickle維護作業,是否運作正常
\item 系統需於每天檢查是否運作正常
\item 於系統完成後,需撰寫技術文件以方便下一位監管者或修改者使用
\end{enumerate}
\section{Performance Requirements}
\begin{enumerate}
\item 每次辨識必須在0.1秒完成(讓real-time辨識得以實現)
\item 準確率(預測結果等於實際值)必須在70\%以上
\end{enumerate}

\section{Safety Requirements}
\begin{enumerate}
\item file防呆(只能選定.png或是.wav)
\item 檢查帳號避免重複註冊
\item 檢查帳密是否正確
\end{enumerate}

\section{Security Requirements}
使用pickle檔存使用者帳密,pickle檔提供了一個簡單的持久化功能。可以將對象以文件的形式存放在磁盤上,用pickle來序列化使得存使用者帳密不會直接洩漏


\end{CJK*}
\end{document}
